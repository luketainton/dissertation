\section{Problem definition}
\label{section:intro-definition}
The Faculty was approached by Multi-Tech Systems and was tasked with designing and building a replacement for their ageing remote access solution using analog modems, namely the MT9234ZBA. Due to the ageing, the hardware that is currently offered to customers has reached the end of its life and is no longer supported, meaning that Multi-Tech support engineers are severely limited in the support they can give to end users if the hardware develops a fault. Some functionality and security concerns were also raised in the project proposal which needed to be fixed. \\\\
It is expected that the end product of this project will be developed further by Multi-Tech and marketed to their customer base as a replacement of, and an alternative to, the current offering. 

\section{Scope}
\label{section:intro-scope}
% This section identifies the boundaries of the project, what was included and what was excluded from the final project. This should be justified and underpinned by research.
\textcolor{red}{TO DO}

\section{Rationale}
\label{section:intro-rationale}
% Why has the topic been chosen?  This may be because of lack of research in the area, to shed more ideas and opinion, in response to a request, e.g. from company or organisation, or a relevant current issue.  It should be more than for personal interest – you should be able to identify a company, organisation or other defined group that will benefit from the work. What benefits can be identified from doing the project. \\\\
This project was given to the University by Multi-Tech Systems, who directly sponsored and supported the research, development, and build phases. As a result of this, the main motivator for the project is commercialisation - the company is seeking to benefit directly from the outcome of the project, likely by marketing the solution on their product catalogue. \\\\
However, there are flaws in the out of band management method that this project is aiming to replace (using modems to 'dial in' to a network device), and this is the secondary motivator of the project. There is a lack of research into the security and functionality issues of using out of band management systems to access network devices, while the way in which modems are designed to operate have security implications when paired with a device such as a Cisco router - something which vendors such as Cisco have had to build protections for.

\section{Project aims and objectives}
\label{section:intro-aims}
\subsection{Aim}
The aim of this project was to design and implement a method of using a Multi-Tech Conduit LoRaWAN Gateway as a Remote Access server, to be used to remotely access the console line of a networking device such as a Cisco router.
\subsection{Objectives}
A list of objectives were identified that needed to be met to ensure that the aim was reached. These are listed below:
\begin{itemize}
    \item Identify the functionality and security issues that exist regarding remote access to devices via a console session.
    \item Evaluate the robustness of the current solution in comparison to the issues identified in the previous objective.
    \item Create the new RAS solution for the Multi-Tech Conduit gateway.
    \item Test the functionality and security of the new solution against the existing product.
\end{itemize}

\section{Background information}
\label{section:intro-background}
\subsection{Multi-Tech Systems}
Multi-Tech Systems manufactures equipment for use cases in the industrial Internet of Things. Founded in 1970, Multi-Tech holds over 80 US patents in technologies ranging from DSL modems to proxy servers. The company has offices in the United States, the United Kingdom, and Japan, allowing it to reach customers from anywhere in the world. Their fast-selling product ranges are in the areas of 4G and LoRaWAN, as companies start to use IoT devices and sensors in their business processes.